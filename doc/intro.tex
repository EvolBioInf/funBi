Bioinformatics is concerned with the computational analysis of
biological data, usually genome data. For example, genomes are
sequenced in the lab, yielding many overlapping sequence fragments
called ``reads''. Assembly of reads into the underlying genome
sequence is a classical bioinformatics task.

In this two hour practical we encounter four of the fundamental
techniques useful in bioinformatics, which are how to
\begin{enumerate}
\item use the Unix command line
\item typeset words and figures
\item align sequences
\item handle genome data
\end{enumerate}
We assume you are now sitting in front of a computer with the Unix
command line running in a terminal. To make the most of what's ahead,
you should be comfortable using a standard keyboard. If you'd like to
test your typing skills, try some of the exercises in a course on
touch typing, for example, \ty{gtypist}.
